\setRecipeMeta{Red Bell Pepper Pasta}{4}{25 MIN}{15 MIN}{./images/redbellpepperpasta.jpg}

\begin{recipe}
  \begin{ingredients}
    \ingredient{500g pasta}
    \ingredient{2 red bell peppers}
    \ingredient{1 big onion}
    \ingredient{3 tbsp olive oil}
    \ingredient{1/2 tsp red pepper flakes}
    \ingredient{2 cloves garlic}
    \ingredient{2 tbsp tomato paste}
    \ingredient{200g heavy cream}
    \ingredient{Parmesan or Pecorino to taste}
    \ingredient{Salt to taste}
    \ingredient{1 tbsp butter}
  \end{ingredients}
  \begin{steps}
    \step{Char the Peppers: Lightly oil peppers and broil until skins are lightly charred, rotating as needed. Transfer to a bowl, cover with plastic wrap, and steam for 15 minutes. Prep remaining ingredients while waiting.}
    \step{Prep Ingredients: Thinly slice onion and garlic. Measure chili flakes.}
    \step{Peel and Slice Peppers: Scrape off pepper skins, remove seeds and stems, then slice thinly.}
    \step{Cook Vegetables: Boil a pot of salted water. In a large pan over medium-high, heat olive oil and add onions with a pinch of salt. Sauté until soft and translucent. Add sliced peppers, then garlic and chili flakes; cook until garlic is slightly browned.}
    \step{Make Sauce: Add 2 tbsp tomato paste, cook for 2-3 minutes. Pour in cream, reduce heat to medium, and simmer until thickened, about 5 minutes. Blend until smooth.}
    \step{Cook Pasta: Add pasta to boiling water. Return sauce to pan on low heat. Melt in 1 tbsp cold butter, adjust seasoning.}
    \step{Combine: Add pasta to sauce 1-2 minutes before al dente, along with a bit of pasta water. Toss until creamy. Remove from heat, stir in Pecorino Romano or Parmesan, and plate with extra cheese.}
  \end{steps}
\end{recipe}
